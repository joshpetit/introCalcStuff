%! Author = joshu
%! Date = 10/05/20

% Preamble
\documentclass[11pt]{article}
\title{Problem Set \# 8}
\author{Joshua Petitma}
\date{10/05/2020}
\renewcommand{\thesubsection}{\thesection.\alph{subsection}}
\newcommand{\bv}[2]{\big\vert_{#1}^{#2}}
\usepackage{xcolor,cancel}

\newcommand\hcancel[2][black]{\setbox0=\hbox{$#2$}%
\rlap{\raisebox{.45\ht0}{\textcolor{#1}{\rule{\wd0}{1pt}}}}#2}
\usepackage{hyperref}
% Packages
\usepackage{amsmath}
\usepackage{amssymb}

% Document
\begin{document}
    \maketitle
    \section[Question 1]{}
    \label{sec:1a}
    \subsection[1.a]{$ \sum_{k=1}^{\infty} \frac{(-1)^k}{k\ln(k)}$}
    \label{subsec:1a}
    \begin{flalign*}
        &\lim_{k\to\infty} \frac{1}{k\ln(k)} = 0 && \\
        &S_n < S_{n+1} \therefore\ Converges\\
        &\frac{1}{kln(k)} < \frac{1}{k} \\
        &\frac{k}{1} * \frac{1}{kln(k)} = \frac{1}{\ln(k)}\\
        &\lim_{k\to\infty} \frac{1}{ln(k)} = 0 \therefore \text{Divergent because of LCT} \\
        &\text{Converges Conditionally}
    \end{flalign*}

    \label{sec:1b}
    \subsection[1.b]{$ \sum_{k=1}^{\infty} (\frac{-4}{5})^k$}
    \label{subsec:1b}
    \begin{flalign*}
        \frac{-4}{5} * \frac{1}{1 + \frac{4}{5}} &&\\
        \frac{4}{5+4} \\
        \box{\frac{4}{9}} \\
        \text{Converges Absolutely}
    \end{flalign*}

    \subsection[1.c]{$ \sum_{k=1}^{\infty} \frac{(-4)^k}{k^2}$}
    \label{subsec:1c}
    \begin{flalign*}
        \lim_{k\to\infty}\frac{(-4)^k}{k^2} \neq 0 \therefore Divergent &&
    \end{flalign*}

    \subsection[1.d]{$ \sum_{k=1}^{\infty} \frac{2 + (-1)^k}{k^2}$}
    \label{subsec:1d}
    \begin{flalign*}
        &\lim_{k\to\infty}\frac{2 + 1}{k^2} && \\
        &\frac{3}{k^2}\ 2 > 1 \therefore\ \text{Convergent because of p-series}\\
        &\frac{2 + (-1)^k}{k^2} \leq \frac{3}{k^2}\ \therefore \text{Converges absolutely bc of CT}
    \end{flalign*}

    \subsection[1.e]{$ (-1)^k\sin(\frac{1}{k})$}
    \label{subsec:1e}
    \begin{flalign*}
        &\lim_{k\to\infty} a_k = 0 &&\\
        &S_n < S_{n+1} \therefore \text{Convergent b/c of AST} \\
        &\sum_{k=1}^{\infty} \sin(\frac{1}{k}) \\
        &\sin(\frac{1}{k})\ \text{Compare too}\ \frac{1}{k} \\
        &\frac{k}{1} * \sin(\frac{1}{k}) \\
        &\lim_{k\to\infty} \frac{sin(\frac{1}{k})}{\frac{1}{k}} \\
        &-\frac{cos(\frac{1}{k}) * -\frac{1}{k^2}}{\frac{1}{k^2}}= -\frac{1}{0} \therefore \text{Converges Conditionally b/c LCT} \\
    \end{flalign*}
    
    \subsection[1.f]{$ \sum_{k=1}^{\infty} (-1)^k k\arctan(\frac{1}{k})$}
    \label{subsec:1f}
    \begin{flalign*}
       &\lim_{k\to\infty} \frac{\tan^{-1}(\frac{1}{k})}{\frac{1}{k}} = \frac{tan^{-1}(0)}{0} = \frac{\pi/2}{0} \therefore\ \text{Divergent bc of test for divergence}
    \end{flalign*}

    \subsection[1.g]{$ (-1)^k \cos(\frac{1}{k})$}
    \label{subsec:1g}
    \begin{flalign*}
        \lim_{k\to\infty}\cos(\frac{1}{k}) = 1\ \therefore\ \text{Divergent bc of limit test}
    \end{flalign*}

    \subsection[1.h]{$ \sum_{k=1}^{\infty} \frac{(-1)^k k}{\sqrt{k^3 -1}}$}
    \label{subsec:1h}
    \begin{flalign*}
        &\lim_{k\to\infty} \frac{(-1)^kk}{\sqrt{k^3 - 1}} = 0  \\
        &S_n < S_n+1\ \therefore\ \text{Converges bc of AST} \\
        &\frac{k}{\sqrt{k^3 - 1}} > \frac{k}{\sqrt{k^3}} \\
        &\frac{1}{\sqrt{k}}\\
        &\frac{\sqrt{k}}{1} * \frac{k}{\sqrt{k^3 - 1}} \\
        &\lim_{k\to\infty}\frac{\sqrt{k^3}}{\sqrt{k^3 - 1}} = 1 \ \therefore \text{Converges Absolutely bc of LCT}
    \end{flalign*}

    \section[Question 2]{$ a_k = \frac{(-1)^k}{k^2 + 1}$}
    \label{sec:2}
    \subsection[2.a]{Show sum converges}
    \label{subsec:2a}
    \begin{flalign*}
        &\lim_{k\to\infty} a_k = 0 \\
        &\frac{(-1)^k}{k^2 + 1} < \frac{1}{k^2}\ \therefore\ \text{Converges bc of comparison test} \\
    \end{flalign*}

    \subsection[2b]{Find error upperbound with 10}
    \label{subsec:2b}
    \begin{flalign*}
        E_{10} < |S_{11}| = \frac{1}{122}
    \end{flalign*}

    \subsection[2.c]{Find smallest value so differences of sum to infty and sum to n is $< \frac{1}{101}$}
    \label{subsec:2c}
    \begin{flalign*}
        a_{n+1} < \frac{1}{101} \\
        \frac{1}{(n+1)^2 + 1} < \frac{1}{101} \\
        \boxed{n \leq 9}\\
    \end{flalign*}

    \subsection[2.d]{Find an upperbound}
    \label{subsec:2d}
    \begin{flalign*}
        E_{10} < \lim_{a\to\infty}\int_{10}^{a} \frac{1}{k^2 + 1} \\
        \tan^{-1}(k) \bv{10}{a} \\
        \boxed{\frac{\pi}{2} - \tan^{-1}(10)} \\
    \end{flalign*}

    \section[Question 3]{}
    \label{sec:3}
    \begin{flalign*}
       &a_k = \frac{(-1)^k}{k^2}:\ E_{100} < \frac{1}{10000} \\
       &b_k = \frac{2}{k^3}:\ E_{100} < \frac{2}{1000000} \\
       &E_{100}\ \text{for}\ a_k < E_{100}\ \text{for}\ b_k\ \therefore\ \text{$ a_k$ aprox. is better}
    \end{flalign*}

    \section[Question 4]{}
    \label{sec:4}
    \subsection[4.a]{}
    \label{subsec:4a}
    \begin{flalign*}
        \int_{11}^{\infty} a_k < E_n < \int_{10}^{\infty} a_k
    \end{flalign*}

    \subsection[4.b]{}
    \label{subsec:4b}
    \begin{flalign*}
       \int_{3}^{\infty} f(x) + \frac{5n}{2n+5} < \int_{2}^{\infty} f(x)
    \end{flalign*}
    Adding the sum brings the lower bound closer to the actual value

    \subsection[4.c]{$ a_k=ke^{-k^2}$}
    \label{subsec:4c}
    \begin{flalign*}
        E_n < \int_{n}^{\infty} xe^{-x^2}dx && u &= -x^2 && \\
         && du &=-2xdx && \\
        -\frac{1}{2} \int e^u du \\
        \lim_{a\to\infty} e^{-k^2} \bv{n}{\infty} \\
        -\frac{1}{2}( 0 - e^{-n^2}) \\
        \frac{1}{2e^{n^2}} \leq \frac{1}{100} \\
        \frac{1}{e^{n^2}} \leq \frac{1}{50} \\
        e^{n^2} \leq 50 \\
        \ln(50) \leq n^2 \\
        \boxed{\sqrt{\ln(50)} \leq n}
%        a_{n} \leq \frac{1}{100} \\
%        (n)e^{-(n^2)} \leq \frac{1}{100} \\
%        e^{-{n^2}} \leq \frac{1}{100n} \\
%        \frac{n}{e^{n^2}} \leq \frac{1}{100} \\
%        \frac{n}{e^{n^2}} \leq \frac{1}{100} \\
%        e^{n^2} \leq 100n \\
%        \ln(100n) \leq n^2 \\
    \end{flalign*}


\end{document}
