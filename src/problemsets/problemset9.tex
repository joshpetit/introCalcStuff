%! Author = joshu
%! Date = 10/11/20

% Preamble
\documentclass[11pt]{article}
\title{Problem Set \# 9}
\author{Joshua Petitma}
\date{10/11/20}
\renewcommand{\thesubsection}{\thesection.\alph{subsection}}
\newcommand{\bv}[2]{\big\vert_{#1}^{#2}}
\usepackage{xcolor,cancel}

\newcommand\hcancel[2][black]{\setbox0=\hbox{$#2$}%
\rlap{\raisebox{.45\ht0}{\textcolor{#1}{\rule{\wd0}{1pt}}}}#2}
\usepackage{hyperref}
% Packages
\usepackage{amsmath}
\usepackage{amssymb}

% Document
\begin{document}
    \maketitle
    \section[Question 1]{}
    \label{sec:1}
    \subsection[1.a]{}
    \label{subsec:1a}
    \begin{flalign*}
        &\frac{a_k}{k} < a_k \therefore \text{Must be Convergent bc of Comparison test} &&\\
    \end{flalign*}

    \subsection[1.b]{$ \sum_{k=1}^{\infty} \sqrt{a_k}$}
    \label{subsec:1b}
    \begin{flalign*}
        &a_k = \frac{1}{k^2} && \frac{1}{\sqrt{k^2}}& && \\
        && \frac{1}{k} &= \text{Divergence bc p-series} &&\\
        &a_k = \frac{1}{k^3} &&  \frac{1}{k^3}& \\
        && \frac{1}{k^{3/2}}&\ \therefore\ \text{Convergent bc p-series}\\
        &\boxed{\text{Impossible to decide}} &&
    \end{flalign*}

    \subsection[1.c]{$ \sum_{k=1}^{\infty} \sin(a_k)$}
    \label{subsec:1c}
    \begin{flalign*}
        &sin(a_k)\ \text{comp to } a_k &&\\
        &\lim_{k\to\infty} \frac{\sin(a_k)}{a_k} \\
        &\lim_{k\to0} \frac{\sin(x)}{x} \to LH\\
        &\cos(x) = 1\ \therefore \text{Must be convergent bc LCT}
    \end{flalign*}

    \subsection[1.d]{$ \cos(a_k)$}
    \label{subsec:1d}
    \begin{flalign*}
        \lim_{k\to\infty} \cos(a_k) = \cos(0) = 1 \therefore \text{Must diverge bc Test for divergence}
    \end{flalign*}
    
    \subsection[1.e]{$ \sum_{k=1}^{\infty} \cos(k)a_k$}
    \label{subsec:1e}
    \begin{flalign*}
        |\cos(k)a_k |\ \leq |a_k| \therefore \text{Converge bc of Comparison test} \\
        \text{Must converge bc of Absolute Convergence Test}
    \end{flalign*}

    \subsection[1.f]{}
    \label{subsec:1f}
    \begin{flalign*}
        &a_k = \frac{1}{k^2} &&\\
        &b_k = \frac{1}{k^2}\\
        &\lim_{k\to\infty} \frac{1/k^2}{1/k^2} = 1 \therefore \text{Diverge because of limit test}\\
        &a_k = \frac{1}{k^4} \\
        &b_k = \frac{1}{k^2} \\
        &\lim_{k\to\infty}\frac{1}{k^4} * \frac{k^2}{1} \\
        &\frac{1}{k^2} \therefore \text{Converge because of p-series}\\
        &\boxed{\text{Impossible to decide}}
    \end{flalign*}

    \subsection[1.g]{$ \arctan(\frac{1}{a_k})$}
    \label{subsec:1g}
    \begin{flalign*}
        &\text{If $a_k$ converges, $ \lim_{k\to\infty}a_k = 0$} &&\\
        &\lim_{k\to\infty}\arctan(\frac{1}{0}) = \arctan(\infty) = \frac{\pi}{2} \neq 0\\
        &\therefore\ \text{Must diverge, test for divergence}
    \end{flalign*}

    \section[Question 2]{}
    \label{sec:2}
    \subsection[2.a]{}
    \label{subsec:2a}
    \begin{flalign*}
        &f(x) = e^{x^2},\ f(0) = 1 &&\\
        &f'(x) = 2x*e^{x^2},\ f'(0) = 0\\
        &f''(x) = 4x^2 * e^{x^2} + 2e^{x^2} \\
        &f''(0) = 2\\
        &T_3 (x) = 1 + \frac{2(x)^2}{2!} = 1 + x^2\\
    \end{flalign*}

    \subsection[2.b]{}
    \label{subsec:2b}
    \begin{flalign*}
        T_3(\frac{1}{10}) &= 1 + (\frac{1}{10})^2 &&\\
        &=1 + \frac{1}{100} = \boxed{\frac{101}{100}}
    \end{flalign*}

    \subsection[2.c]{}
    \subsection[2.c]{}
    \label{subsec:2c}
    \begin{flalign*}
        &\int_{0}^{\frac{1}{10}} T_3 (x)dx && \\
        &\int_{0}^{\frac{1}{10}} 1 + x^2 dx \\
        &(x + \frac{x^3}{3})\bv{0}{\frac{1}{10}} \\
        &(\frac{1}{10} + \frac{1}{3000}) - 0 = \frac{301}{3000}
    \end{flalign*}



    \section[Question 3]{}
    \label{sec:3}
    \subsection[3.a]{}
    \label{subsec:3a}
    \begin{flalign*}
        &f(x),\ f(1) = 0 &&\\
        &f'(x) = \frac{1}{x} * x + \ln(x),\ f'(1) = 1 + 0 = 1 \\
        &f''(x) = \frac{1}{x},\ f''(1) = 1\\
        &f'''(x) = -\frac{1}{x^2},\ f'''(1) = -1 \\
        &\boxed{T_3(x) = 0 + 1(x-1) + \frac{1 (x-1)^2}{2!} + -\frac{1(x-1)^3}{3!}}
    \end{flalign*}

    \subsection[3.b]{}
    \label{subsec:3b}
    \begin{flalign*}
        &f(x),\ f(1) = \sqrt{2} &&\\
        &f'(x) = \frac{1}{2}(1+x)^{-1/2},\ f'(1) = \frac{1}{2\sqrt{2}}\\
        &f''(x) = -\frac{1}{4}(1 + x)^{-3/2},\ f''(1) = -\frac{1}{4\sqrt{8}}\\
        &f'''(x) = \frac{3}{8}(1 + x)^{-5/2},\ f'''(1) = \frac{3}{8\sqrt{32}}\\
        &\boxed{T_3(x) = \sqrt{2} + \frac{\frac{1}{2\sqrt{2}}(x-1)}{1!} - \frac{\frac{1}{4\sqrt{8}}(x-1)^2}{2!}
        + \frac{\frac{3}{8\sqrt{32}}(x-1)^3}{3!}} \\
    \end{flalign*}

    \subsection[3.c]{}
    \label{subsec:3c}
    \begin{flalign*}
        &f(x),\ f(\pi/4) = \tan(\pi/4) = 1 &&\\
        &f'(x) =\ \sec^2(x),\ f''(\pi/4) =\sec^2(\pi/4) = 1 \\
        &f''(x) = 2\sec(x) * sec(x)tan(x),\ f''(\pi/4) = 2 \\
        &T_2(x) = 1 + 1 (x - \frac{\pi}{4}) + \frac{2 (x - \frac{\pi}{4})^2}{2!} \\
        &T_2(x) = 1+ (x - \frac{\pi}{4}) + (x - \frac{\pi}{4})^2
    \end{flalign*}

    \section[Question 4]{}
    \label{sec:4}
    \subsection[4.a]{}
    \label{subsec:4a}
    \begin{flalign*}
        &f(-1) = T(-1) = 0 \ngtr 0 &&\\
        &\therefore \text{Impossible bc T(-1) is not positive}&&\\
    \end{flalign*}

    \subsection[4.b]{}
    \label{subsec:4b}
    \begin{flalign*}
        &f(-1) = T(-1) = 1 > 0 &&\\
        &T'(x) = 4 = 4(x+1) \\
        &T''(x) = 4 \\
        &T''(-1) = 4 > 0 \\
        &\therefore\ \text{Possible because T(-1) and T''(-1) are positive}
    \end{flalign*}

    \subsection[4.c]{}
    \label{subsec:4c}
    \begin{flalign*}
        &f(-1) = T(-1) = 9 > 0 &&\\
        &T'(x) = 4(x-1) \\
        &T''(x) = 4 > 0 \\
        &\therefore\ \text{Possible because T(-1) and T''(-1) are positive}
    \end{flalign*}

    \subsection[4.d]{}
    \label{subsec:4d}
    \begin{flalign*}
        &T(-1) = -\frac{1}{6} +2 - 10 = -\# \ngtr 0 &&\\
        &\therefore\ \text{Impossible because T(-1) is not positive}
    \end{flalign*}

    \section[Question 5]{}
    \label{sec:5}
    \subsection[5.a]{}
    \label{subsec:5a}
    \begin{flalign*}
        x && f(x) && f'(x) && f''(x) && f'''(x) && f^4(x) \\
        0 && X && X && X && X && X \\
        1 && 1 && 2 && 2 && 0 && X \\
        2 && 7 && 3 && 1 && 0 && 48
    \end{flalign*}

    \subsection[5.b]{}
    \label{subsec:5b}
    \begin{flalign*}
        &u = f''(x) &&\\
        &du = f'''(x) \\
        &\int_{2}^{1} f''(u)du && f''(1) = 2,\ f''(2) = 1 \\
        &- (f(u) \bv{1}{2}) \\
        &-(f(2) - f(1)) \\
        &-(7 - 1) = \boxed{-6}
    \end{flalign*}

    \subsection[5.c]{}
    \label{subsec:5c}
    \begin{flalign*}
        &\lim_{x\to0} \frac{f(1) - 1}{f'''(2)} = \frac{0}{0} \to LH &&\\
        &\lim_{x\to0} \frac{f'(e^x) - 1}{f^4(2x + 2)} = \frac{2 - 1}{48} = \boxed{\frac{1}{48}}
    \end{flalign*}

    \subsection[5.d]{}
    \label{subsec:5d}
    \begin{flalign*}
        &g(x) = that,\ g(1) = e^{(f(1)^2)} = e^1 &&\\
        &g'(x) = 2f(x) * f'(x) * e^{f(x)^2},\ g'(1) = 2*2*e = 4e \\
        &T_1(x) = e + 4e (x - 1)
    \end{flalign*}

    \subsection[5.e]{}
    \label{subsec:5e}
    \begin{flalign*}
        &h(x) = that,\ h(1) = f(1) = 1 &&\\
        &h'(x) = f'(1 + sin(\pi x)),\ h'(1) = f'(1) = 2 \\
        &h''(x) = f''(1 + \sin(\pi x)), \h''(1) = f''(1) = 2\\
        &T_2(x) = 1 + 2(x-1) + \frac{2(x-1)^2}{2!} \\
        &T_2(x) = 1 + 2(x-1) + (x-1)^2 \\
    \end{flalign*}


\end{document}
