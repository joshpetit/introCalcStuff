%! Author = joshu
%! Date = 9/8/20

% Preamble
\documentclass[11pt]{article}
\renewcommand{\thesubsection}{\thesection.\alph{subsection}}
\newcommand{\bv}[2]{\big\vert_{#1}^{#2}}
\usepackage{xcolor,cancel}

\newcommand\hcancel[2][black]{\setbox0=\hbox{$#2$}%
\rlap{\raisebox{.45\ht0}{\textcolor{#1}{\rule{\wd0}{1pt}}}}#2}
\usepackage{hyperref}
% Packages
\usepackage{amsmath}
\usepackage{amssymb}

% Document
\begin{document}
    \section[Question 1]{}
    \label{sec:1}
    \subsection[1.a]{}
    \label{subsec:1a}

    \subsubsection[1.a.1]{Find the following probabilities}
    \label{subsubsec:1a1}
    \begin{itemize}
        \item $P(A) = \frac{5}{20}$
        \item $P(B) = \frac{10}{20}$
        \item $P(A \bigcap B) = \frac{2}{20} = \frac{1}{10}$
    \end{itemize}

    \subsubsection[1.a.2]{Are A and B independent}
    \label{subsec:1a2}
    $P(A \bigcap B) = \frac{1}{10} \neq \frac{5}{20} * \frac{10}{20} \therefore$ No, they are not independent

    \subsection[1.b]{Roll 3 times}
    \label{subsec:1b}
    \begin{flalign*}
        4(DDD), 16(DDN), 16(DNN), 16(NNN) &&\\
        P(X) = \frac{4}{20} &&\\
        p(0) = \frac{3}{5} * \frac{1}{8} = \frac{3}{40} \\
        p(1) = \frac{3}{5} * \frac{3}{8} = \frac{9}{40} \\
        p(2) = \frac{3}{5} * \frac{3}{8} = \frac{9}{40} \\
        p(3) = \frac{3}{5} * \frac{1}{8} = \frac{3}{40} \\
    \end{flalign*}
    \section[Question 2]{}
    \label{sec:2}
    \subsection[2.a]{}
    \label{subsec:2a}
    \begin{flalign*}
        \sum_{k=1}^{n} x_{kp(x_k)} && \\
        p(Y = -2 \bigcup -1) = \frac{1}{2} \\
        \frac{1}{10} + a = \frac{1}{2} \\
        a = \frac{4}{10} \\
        -2(\frac{1}{10}) - \frac{4}{10} + 0(b) + 1(\frac{3}{10}) + 2(c) = 0 \\
        -\frac{2}{10} - \frac{4}{10} + 0(b) + \frac{3}{10} + 2(c) = 0 \\
        - \frac{6}{10} + \frac{3}{10} + 2(c) = 0 \\
        - \frac{3}{10} + 2(c) = 0 \\
        c = \frac{3}{20} \\
        b = 1 - \frac{4}{10} - \frac{3}{10} - \frac{1}{10} - \frac{3}{20} \\
        b = 1 - \frac{8}{10} - \frac{3}{20} \\
        b = 1 - \frac{16}{20} - \frac{3}{20} \\
        b = 1 - \frac{19}{20} = \frac{1}{20}
    \end{flalign*}
    \subsection[2.b]{Find $E[Y^2]$}
    \label{subsec:2b}
    \begin{flalign*}
        -2(\frac{1}{10}) - \frac{4}{10} + 0(b) + 1(\frac{3}{10}) + 2(c) = 0 \\
    \end{flalign*}

    \section[Question 3]{}
    \label{sec:3}
    \subsection[3.a]{}
    \label{subsec:3a}
    \subsubsection[3.a.1]{}
    \label{subsubsec:3a1}
    \begin{flalign*}
        p(1) = \frac{c}{1} &&\\
    \end{flalign*}

    \section[Question 4]{}
    \label{sec:4}
    \subsection[4.a]{$a_1$}
    \label{subsec:4a}
    \begin{flalign*}
        \frac{1}{2(1) + 4} = \frac{1}{6} && \\
    \end{flalign*}
    \subsection[4.b]{$ \sum_{k=1}^{\infty} a_k$}
    \label{subsec:4b}
    \begin{flalign*}
        \lim_{a\to\infty}\int_{1}^{a} 1 - \frac{n}{2n+4}dn && \\
        \lim_{a\to\infty}n \bv{1}{a} - \int_{1}^{a} \frac{n}{4(\frac{n}{2} + 1)}
    \end{flalign*}
    \subsection[4.c]{$ \lim_{k\to\infty}$}
    \label{subsec:4c}
    \begin{flalign*}
        \lim_{n\to\infty} \frac{n}{2n + 4} = \frac{1}{2}&& \\
        1 - \frac{1}{2} = \frac{1}{2}
    \end{flalign*}
    \subsection[4.d]{$ \sum_{k=101}^{200} a_k$}
    \label{subsec:4d}
    \begin{flalign*}
    \end{flalign*}

    \subsection[4.e]{$a_{100}$}
    \label{subsec:4e}
    \begin{flalign*}
        1 - \frac{100}{200 + 4} = 1 - \frac{100}{204} && \\
    \end{flalign*}

    \section[Question 5]{}
    \label{sec:5}
    \subsection[5.a]{}
    \label{subsec:5a}
    \begin{flalign*}
        \sum_{k=1}^{\infty} (9^{\frac{1}{k}} - 9^{\frac{1}{k+2}}) && \\
        S_1 &= 9^1 - 9^{\frac{1}{3}} \\
        S_2 &= (9^1 - 9^{\frac{1}{3}}) + (9^{\frac{1}{2}} - 9^{\frac{1}{4}}) \\
        S_3 &= (9^1 - \hcancel[red]{9^{\frac{1}{3}}}) + (9^{\frac{1}{2}} - 9^{\frac{1}{4}}) + (\hcancel[red]{9^{\frac{1}{3}}} - 9^{\frac{1}{5}}) \\
        S_3 &= (9^1 ) + (3 - 9^{\frac{1}{4}}) + (- 9^{\frac{1}{5}}) \\
        S_n &= 9 + 3 - 9^{\frac{1}{n+1}} - 9^{\frac{1}{n+2}} \\
        S_n &= 12 - 9^{\frac{1}{n+1}} - 9^{\frac{1}{n+2}} \\
        \lim_{n\to\infty} S_n &= 12 - 9^0 - 9^0 = 10
    \end{flalign*}
    \subsection[5.b]{$\sum_{k=1}^{\infty} e^{\sin(\frac{1}{k})} && \\$}
    \label{subsec:5b}
    \begin{flalign*}
        \lim_{k\to\infty} e^{\sin(0)} && \\
        \lim_{k\to\infty} e^0 = 1 \neq 0  &\therefore\ \text{Divergent because of limit test}
    \end{flalign*}
    \subsection[5.c]{$\sum_{k=1}^{\infty}\frac{1+e^{-k}}{4e^{-k}+3}$}
    \label{subsec:5c}
    \begin{flalign*}
        \lim_{k\to\infty} \frac{1 + e^{-k}}{4e^{-k} + 3} && \\
        \lim_{k\to\infty}\frac{1}{4 + 3} = \frac{1}{7} \neq 0 & \therefore\ \text{Divergent because of limit test}
    \end{flalign*}
    \subsection[5.d]{$ \sum_{k=1}^{\infty}(\cos(\frac{\pi}{k}) - \cos(\frac{\pi}{k+2}))$}
    \label{subsec:5d}
    \begin{flalign*}
        S_1 &= \cos(\frac{\pi}{1}) - \cos(\frac{\pi}{3}) && \\
        S_2 &= (\cos(\frac{\pi}{1}) - \cos(\frac{\pi}{3})) + (\cos(\frac{\pi}{2}) - \cos(\frac{\pi}{4})) \\
        S_3 &= (\cos(\frac{\pi}{1}) - \hcancel[red]{\cos(\frac{\pi}{3})}) + (\cos(\frac{\pi}{2}) - \cos(\frac{\pi}{4}))
        + (\hcancel[red]{\cos(\frac{\pi}{3}}) - \cos(\frac{\pi}{5})) \\
        S_3 &= (\cos(\frac{\pi}{1})) + (\cos(\frac{\pi}{2}) - \cos(\frac{\pi}{4})) - \cos(\frac{\pi}{5})) \\
        S_n &= -1 + 0 - \cos(\frac{\pi}{n+1}) - \cos(\frac{\pi}{n+2}) \\
        S_n &= -1 - \cos(\frac{\pi}{n+1}) - \cos(\frac{\pi}{n+2}) \\
        \lim_{n\to\infty} &= -1 - \cos(0) - \cos(0) = -1 -1 -1 = -3\\
    \end{flalign*}

\end{document}