%! Author = joshu
%! Date = 9/8/20

% Preamble
\documentclass[11pt]{article}
\title{Problem Set \# 6}
\author{Joshua Petitma}
\date{09/24/20}
\renewcommand{\thesubsection}{\thesection.\alph{subsection}}
\newcommand{\bv}[2]{\big\vert_{#1}^{#2}}
\usepackage{xcolor,cancel}

\newcommand\hcancel[2][black]{\setbox0=\hbox{$#2$}%
\rlap{\raisebox{.45\ht0}{\textcolor{#1}{\rule{\wd0}{1pt}}}}#2}
\usepackage{hyperref}
% Packages
\usepackage{amsmath}
\usepackage{amssymb}

% Document
\begin{document}
    \maketitle
    \section[Question 1]{}
    \label{sec:1}
    \subsection[1.a]{}
    \label{subsec:1a}

    \subsubsection[1.a.1]{Find the following probabilities}
    \label{subsubsec:1a1}
    \begin{itemize}
        \item $P(A) = \frac{5}{20}$
        \item $P(B) = \frac{10}{20}$
        \item $P(A \bigcap B) = \frac{2}{20} = \frac{1}{10}$
    \end{itemize}

    \subsubsection[1.a.2]{Are A and B independent}
    \label{subsec:1a2}
    $P(A \bigcap B) = \frac{1}{10} \neq \frac{5}{20} * \frac{10}{20} \therefore$ No, they are not independent

    \subsection[1.b]{Roll 3 times}
    \label{subsec:1b}
    \subsubsection[1.b.1]{Mass Density function}
    \begin{flalign*}
        (DDD), 3 * (DDN), 3 * (DNN), (NNN) &&\\
        P(X) = \frac{4}{20} &&\\
        p(0) = (\frac{16}{20})^3 = (\frac{4}{5})^3 = \frac{64}{125}\\
        3 * p (1) = \frac{4}{20}(\frac{4}{5})^2 = \frac{1}{5}(\frac{4}{5})^2 = \frac{1}{5} * \frac{16}{12} = \frac{16}{125} * 3 = \frac{48}{25}\\
        3 * p(2) = (\frac{1}{5})^{2} * \frac{4}{5} = \frac{1}{25} * \frac{4}{5} = \frac{4}{125} * 3 = \frac{12}{125}\\
        p(3) = (\frac{1}{5})^3  = \frac{1}{125}\\
    \end{flalign*}
    \subsubsection[1.b.2]{Expected value}
    \begin{flalign*}
        0 * \frac{64}{125} + 1 * \frac{48}{125} + 2 * \frac{12}{125} + 3 * \frac{1}{125} && \\
        \frac{48}{125} + \frac{24}{125} + \frac{3}{125} = \frac{75}{125} = \frac{3}{5}
    \end{flalign*}
    \section[Question 2]{}
    \label{sec:2}
    \subsection[2.a]{}
    \label{subsec:2a}
    \begin{flalign*}
        \sum_{k=1}^{n} x_{kp(x_k)} && \\
        p(Y = -2 \bigcup -1) = \frac{1}{2} \\
        \frac{1}{10} + a = \frac{1}{2} \\
        a = \frac{4}{10} \\
        -2(\frac{1}{10}) - \frac{4}{10} + 0(b) + 1(\frac{3}{10}) + 2(c) = 0 \\
        -\frac{2}{10} - \frac{4}{10} + 0(b) + \frac{3}{10} + 2(c) = 0 \\
        - \frac{6}{10} + \frac{3}{10} + 2(c) = 0 \\
        - \frac{3}{10} + 2(c) = 0 \\
        c = \frac{3}{20} \\
        b = 1 - \frac{4}{10} - \frac{3}{10} - \frac{1}{10} - \frac{3}{20} \\
        b = 1 - \frac{8}{10} - \frac{3}{20} \\
        b = 1 - \frac{16}{20} - \frac{3}{20} \\
        b = 1 - \frac{19}{20} = \frac{1}{20}
    \end{flalign*}
    \subsection[2.b]{Find $E[Y^2]$}
    \label{subsec:2b}
    \begin{flalign*}
        4(\frac{1}{10}) + \frac{4}{10} + 0 + 1(\frac{3}{10}) + 4(\frac{3}{20}) = 0 \\
        \frac{4}{10} + \frac{4}{10} + \frac{3}{10} + \frac{3}{5} \\
        \frac{11}{10} + \frac{6}{10} = \frac{17}{10}
    \end{flalign*}

    \section[Question 3]{}
    \label{sec:3}
    \subsection[3.a]{}
    \label{subsec:3a}
    \subsubsection[3.a.1]{}
    \label{subsubsec:3a1}
    \begin{flalign*}
        p(1) = \frac{c}{1} &&\\
        p(2) = \frac{c}{2} \\
        p(3) = \frac{c}{3} \\
        p(4) = \frac{c}{4} \\
        p(5) = \frac{c}{5} \\
        p(6) = \frac{c}{6} \\
    \end{flalign*}
    \subsubsection[3.a.2]{Roll an even}
    \begin{flalign*}
        \frac{c}{2} + \frac{c}{4} + \frac{c}{6} &&\\
        \frac{6c}{12} + \frac{3c}{12} + \frac{2c}{12} = \frac{11c}{12}
    \end{flalign*}
    \subsubsection[3.a.3]{$\frac{1}{X} < \frac{1}{5}$}
    \begin{flalign*}
        S = \{ \frac{1}{6} \} && \\
        p(6) = \frac{c}{6} \\
    \end{flalign*}

    \subsubsection[3.a.4]{$P(X^2 - 4 < 5)$}
    \begin{flalign*}
        X = 1, 2 && \\
        P(1 \bigcup 2) &= \frac{c}{1} + \frac{c}{2}\\
        P(1 \bigcup 2) &= \frac{3c}{2}
    \end{flalign*}

    \subsubsection[3.a.5]{$E[X]$}
    \begin{flalign*}
        1*p(1) + 2*p(2) + \dots \\
        1 * \frac{c}{1} + \hcancel{2} * \frac{c}{\hcancel{2}} + \dots \\
        c + c + c + c + c + c = 6c && \\
    \end{flalign*}

    \subsection[3.b]{Roll twice, $P(2X \leq 2Y - 5)$}
    \label{subsec:3b}
    \begin{flalign*}
        p(6)p(3) +p(6)p(2) +  p(6)p(1) +  p(4)p(1) +p(5)p(1) + p(5)p(2) &&\\
        (\frac{c}{6} * \frac{c}{3} )+(\frac{c}{6} * \frac{c}{2} )+(\frac{c}{6} * c )
        +(\frac{c}{4} * c )+(\frac{c}{5} * c )+(\frac{c}{5} * \frac{c}{2}) &&\\
        (\frac{c^2}{18}) + (\frac{c^2}{12}) + (\frac{c^2}{6}) +  (\frac{c^2}{4}) + (\frac{c^2}{5}) + (\frac{c^2}{10}) \\
        \frac{77c^2}{90}
    \end{flalign*}
    \subsection[3.c]{Find the value of c}
    \label{subsec:3c}
    \begin{flalign*}
        c (1 + \frac{1}{2} + \frac{1}{3} + \frac{1}{4} + \frac{1}{5} + \frac{1}{6}) &= 1 && \\
        c(\frac{49}{20}) &= 1 \\
         c &= \frac{20}{49}
    \end{flalign*}

    \section[Question 4]{}
    \label{sec:4}
    \subsection[4.a]{$a_1$}
    \label{subsec:4a}
    \begin{flalign*}
        \frac{1}{2(1) + 4} = \frac{1}{6} && \\
    \end{flalign*}
    \subsection[4.b]{$ \sum_{k=1}^{\infty} a_k$}
    \label{subsec:4b}
    \begin{flalign*}
        \lim_{n\to\infty} \frac{n}{2n + 4} = \frac{1}{2}&& \\
        1 - \frac{1}{2} = \frac{1}{2}
    \end{flalign*}
    \subsection[4.c]{$ \lim_{k\to\infty}$}
    \label{subsec:4c}
    \begin{flalign*}
        \lim_{k\to\infty} a_k = 0 \text{ because $ \sum_{}^{}a_k$ exists} &&
    \end{flalign*}
    \subsection[4.d]{$ \sum_{k=101}^{200} a_k$}
    \label{subsec:4d}
    \begin{flalign*}
        \sum_{101}^{200}a_k = S_{200} - S_{100} && \\
        \sum_{101}^{200}a_k = 1 - \frac{200}{404} + 1 - \frac{100}{204} \\
    \end{flalign*}

    \subsection[4.e]{$a_{100}$}
    \label{subsec:4e}
    \begin{flalign*}
        1 - \frac{100}{200 + 4} = 1 - \frac{100}{204} && \\
    \end{flalign*}

    \section[Question 5]{}
    \label{sec:5}
    \subsection[5.a]{}
    \label{subsec:5a}
    \begin{flalign*}
        \sum_{k=1}^{\infty} (9^{\frac{1}{k}} - 9^{\frac{1}{k+2}}) && \\
        S_1 &= 9^1 - 9^{\frac{1}{3}} \\
        S_2 &= (9^1 - 9^{\frac{1}{3}}) + (9^{\frac{1}{2}} - 9^{\frac{1}{4}}) \\
        S_3 &= (9^1 - \hcancel[red]{9^{\frac{1}{3}}}) + (9^{\frac{1}{2}} - 9^{\frac{1}{4}}) + (\hcancel[red]{9^{\frac{1}{3}}} - 9^{\frac{1}{5}}) \\
        S_3 &= (9^1 ) + (3 - 9^{\frac{1}{4}}) + (- 9^{\frac{1}{5}}) \\
        S_n &= 9 + 3 - 9^{\frac{1}{n+1}} - 9^{\frac{1}{n+2}} \\
        S_n &= 12 - 9^{\frac{1}{n+1}} - 9^{\frac{1}{n+2}} \\
        \lim_{n\to\infty} S_n &= 12 - 9^0 - 9^0 = 10
    \end{flalign*}
    \subsection[5.b]{$\sum_{k=1}^{\infty} e^{\sin(\frac{1}{k})} \\$}
    \label{subsec:5b}
    \begin{flalign*}
        \lim_{k\to\infty} e^{\sin(0)} && \\
        \lim_{k\to\infty} e^0 = 1 \neq 0  &\therefore\ \text{Divergent because of limit test}
    \end{flalign*}
    \subsection[5.c]{$\sum_{k=1}^{\infty}\frac{1+e^{-k}}{4e^{-k}+3}$}
    \label{subsec:5c}
    \begin{flalign*}
        \lim_{k\to\infty} \frac{1 + e^{-k}}{4e^{-k} + 3} && \\
        \lim_{k\to\infty}\frac{1}{4 + 3} = \frac{1}{7} \neq 0 & \therefore\ \text{Divergent because of limit test}
    \end{flalign*}
    \subsection[5.d]{$ \sum_{k=1}^{\infty}(\cos(\frac{\pi}{k}) - \cos(\frac{\pi}{k+2}))$}
    \label{subsec:5d}
    \begin{flalign*}
        S_1 &= \cos(\frac{\pi}{1}) - \cos(\frac{\pi}{3}) && \\
        S_2 &= (\cos(\frac{\pi}{1}) - \cos(\frac{\pi}{3})) + (\cos(\frac{\pi}{2}) - \cos(\frac{\pi}{4})) \\
        S_3 &= (\cos(\frac{\pi}{1}) - \hcancel[red]{\cos(\frac{\pi}{3})}) + (\cos(\frac{\pi}{2}) - \cos(\frac{\pi}{4}))
        + (\hcancel[red]{\cos(\frac{\pi}{3}}) - \cos(\frac{\pi}{5})) \\
        S_3 &= (\cos(\frac{\pi}{1})) + (\cos(\frac{\pi}{2}) - \cos(\frac{\pi}{4})) - \cos(\frac{\pi}{5})) \\
        S_n &= -1 + 0 - \cos(\frac{\pi}{n+1}) - \cos(\frac{\pi}{n+2}) \\
        S_n &= -1 - \cos(\frac{\pi}{n+1}) - \cos(\frac{\pi}{n+2}) \\
        \lim_{n\to\infty} &= -1 - \cos(0) - \cos(0) = -1 -1 -1 = -3\\
    \end{flalign*}

\end{document}