%! Author = joshu
%! Date = 9/15/20

% Preamble
\documentclass[11pt]{article}

\renewcommand{\thesubsection}{\thesection.\alph{subsection}}
\newcommand{\bv}[2]{\big\vert_{#1}^{#2}}
\usepackage{hyperref}
% Packages
\usepackage{amsmath}

% Document
\begin{document}
    \section[Question 1]{If we roll a fair die, waht is the probability that after 6 rolls we:}
    \label{sec:1}
    \subsection[1.a]{Do not get a 6?}
    \label{subsec:1a}
    \begin{flalign*}
        P(A^c) &= (\frac{5}{6})^6 && \\
    \end{flalign*}
    \subsection[1.b]{Get a 6 on the first roll, but not after?}
    \label{subsec:1b}
    \begin{flalign*}
        P(A) = \frac{1}{6} * (\frac{5}{6})^5 &&
    \end{flalign*}
    \subsection[1.c]{Get Exactly one 6}
    \label{subsec:1c}
    \begin{flalign*}
        P(A) = (\frac{5}{6})^5 &&
    \end{flalign*}
    \section[Question 2]{Suppose you roll 2 fair six sided dice.}
    \label{sec:2}
    \subsection[2.a]{Find P(A) and P(B)}
    \label{subsec:2a}
    \begin{flalign*}
        P(A) = \frac{15}{36} \\
        P(B) = \frac{8}{36} = \frac{2}{9}
    \end{flalign*}
    \subsection[2.b]{}
    \label{subsec:2b}
    \begin{flalign*}
        P(A \bigcap B)&= P(A) * P(B) \\
        P(A \bigcap B) &= \frac{2}{36} = \frac{1}{18} \\
        P(A) * P(B) &= \frac{15}{36} * \frac{2}{9} \\
        P(A \bigcap B) &\neq \frac{5}{54}\\
        \text{Not independent}
    \end{flalign*}
    \subsection[2.c]{}
    \label{subsec:2c}
    \begin{flalign*}
        P(A \bigcup B \text{ 25 \% of time}) = \frac{19}{36} * P(A^c\bigcup B^c)\\
        4*\frac{19}{36} * (\frac{17}{36})^3
    \end{flalign*}

    \section[Question 3]{}
    \label{sec:3}
    \subsection[3.a]{}
    \label{subsec:3a}
    \begin{flalign*}
        S = \{TTT, TTH, THH, HHH \} \\
        X = \{-6, 1 - 4, 4-2, 9 \} \\
        X = \{-6, -3, 2, 9 \} \\
    \end{flalign*}
    \subsection[3.b]{Find probability mass density}
    \label{subsec:3b}
    \begin{flalign*}
        p(-6) = \frac{1}{8} * -6 = \frac{-6}{8}\\
        p(-3) = \frac{3}{8} * -3 = \frac{-9}{8}\\
        p(2) = \frac{3}{8}  * 2 = \frac{6}{8}\\
        p(9) = \frac{1}{8} * 9 = \frac{9}{8}\\
        E = 0
    \end{flalign*}
    \subsection[3.c]{E}
    \label{subsec:3c}
    \begin{flalign*}
        E(x) = \frac{-6}{8} -  \frac{-9}{8} +  \frac{6}{8}
        +  \frac{9}{8} = 0
    \end{flalign*}
    \section[Question 4]{}
    \label{sec:4}
    \subsection[4.a]{}
    \label{subsec:4a}
    \begin{flalign*}
        S = \{3, 4, 5\} \\
        p(3) = \frac{1}{10} * \frac{2}{9} = \frac{2}{90} \\
        p(4) = \frac{8}{45}
    \end{flalign*}





\end{document}