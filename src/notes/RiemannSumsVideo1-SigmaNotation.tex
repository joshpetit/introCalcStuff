%! Author = joshu
%! Date = 9/11/20

% Preamble
\documentclass[11pt]{article}

% Packages
\usepackage{amsmath}
\title{Sigma Notation}
\renewcommand{\thesubsection}{\thesection.\alph{subsection}}
\usepackage{hyperref}
% Document
\begin{document}
    \maketitle
    \begin{flalign*}
        \sum_{1}^{n} a_i = a_1 + a_2 + a_3 + \mathellipsis + a_n \\
        i = \text{The index} &&  \\
        a_i = \text{The $i^{th}$ term} &&
    \end{flalign*}
    \section[Question 1]{}
    \begin{flalign*}
    \label{subsec:1}
        \sum_{1}^{5} i = 1 + 2 + 3 + 4 + 5 && \\
    \end{flalign*}
    \section[Question 2]{}
    \begin{flalign*}
        \sum_{1}^{4} k^3 = 1^3 + 2^3 + 3^3 + 4^3 && \\
    \end{flalign*}
    \section[Question 3]{}
    \begin{flalign*}
        \sum_{1}^{4} 3^k = 3^1 + 3^2 + 3^3 + 3^4 && \\
    \end{flalign*}
    \section[Question 4]{}
    \begin{flalign*}
        \sum_{6}^{8} n^2 = 6^2 + 7^2 + 8^2 && \\
    \end{flalign*}
    \section[Question 5]{}
    \begin{flalign*}
        \sum_{1}^{4} (-1) = (-1)^1 (-1)^2 + (-1)^3 + (-1)^4 && \\
    \end{flalign*}
    Useful notation when we have a long sum \\
    $ \sum_{1}^{100} i = 1 + 2 + 3 + \mathellipsis$
    or sums of variable length ($ \sum_{i}^{n} i = 1 + 2 + 3 $)
    \\
    \section{Properties}
    \begin{enumerate}
        \item 
        \begin{flalign*}
            \sum_{i}^{n} ca_i = c \sum_{i}^{} a_i && \\
            \text{EX:}
            \sum_{i}^{n} 2i = 2(1) + 2(2) +\mathellipsis + 2(n) \\
            = 2 (1 + 2 + \mathellipsis + n) \\
            = 2 \sum_{i }^{n} i
        \end{flalign*}
        \item
        \begin{flalign*}
            \sum_{i }^{n} a=i + b_i &= \sum_{i }^{n} a_i + \sum_{i}^{b} b_i \\
            \\
            \sum_{k=1}^{n}(k + k^2)  &= (1 + 1^2) + (2+2^2) + \mathellipsis + (n^2 +n^2) \\
            &= \sum_{i=1}^{n} i + \sum_{i=1}^{n} i^2
        \end{flalign*}
    \end{enumerate}


\end{document}