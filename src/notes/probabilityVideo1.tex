%! Author = joshu
%! Date = 9/14/20

% Preamble
\documentclass[11pt]{article}

\renewcommand{\thesubsection}{\thesection.\alph{subsection}}
\newcommand{\bv}[2]{\big\vert_{#1}^{#2}}
% Packages
\usepackage{hyperref}
\usepackage{amsmath}

% Document
\begin{document}
    \part[Introduction to Probability]{Introduction to Probability}
    \label{null:introduction-to-probability}
    \section[Definitions]{Definitions}
    \begin{enumerate}
        \item Experiment - A well defined procedure (rolling a die, flipping a coin)
        \item Sample Space - The set of all possible outcomes for particular experiment
        \item Event - A subset of the sample space, subset of outcomes possible
    \end{enumerate}

    \paragraph[Example]{Example: } Rolling a six-sided fair die

    \begin{itemize}
        \item Sample Space: $S = \{1, 2, 3, 4, 5, 6\}$
        \item Possible Events:
        \subitem \textmd{A = ``you roll an even number'' A = \{2, 4, 6\}\\
        P(A) = 3/6 = 1/2}
        \subitem \textmd{B = ``you roll a \# less than 5" \to B = \{1,2,3,4\}
        \subitem $P(B) = 4/6 = 2/3 $}
    \end{itemize}
    \section[Principles]{Principles}
    \label{sec:principles}
    \begin{enumerate}
        \item $0\leq P(event) \leq 1$
        \item $\sum P(outcome) = 1$ The sum of all the outcomes must equal 1,
        Something has to happen
    \end{enumerate}

    \paragraph[Example]{Example: } Suppose a 4 sided die has sides A, B, C, and D.
    You roll it twice
    \begin{enumerate}
        \item Find the sample space \\
        $S = \{AA, AB, AC, AD, BA, BB, BC, BD, CA, CB, CC, CD, DA, DB, DC, DD\}$
        \item Find the following probabilities:
        \subitem P(you roll 2 As) = $\frac{1}{16}$
        \subitem P($1^{st}$ roll is A) = $\frac{1}{4}$
        \subitem P(at least one roll is A) = $\frac{7}{16}$
        \subitem P(roll at least on A or at least one B) = $\frac{12}{16} = \frac{3}{4}$
    \end{enumerate}

    \part[Probability Rules and Independence]{Probability Rules and Independence}
    \section[Rules]{Rules}
    \begin{itemize}
        \item Addition Rule: P(A or $\bigcup$ B) = P(A) + P(B) - P(A and $\bigcap$ B)

        We subtract the intersection (P $\bigcap$ B)because A and B are already counted.
        Think of a Ven diagram
        \begin{itemize}
            \item P(you roll at least one A or at least one B)
            \subitem P(you roll at least one A) + P (you roll at least one B)
            \\ AA, AB, AC, AD, BA, CA, DA +BA, BB, BC, BD, AB, CB, DB
            \item P(you roll at least one A and at least one B)
            \item P (you roll at least one A or at least on B)
        \end{itemize}
        \item Complement Rule: P($A^c$) = 1 - P(A) \\
        $X^C$ = All outcomes that do not satisfy X
        \\
        Ex: P(you roll at least on A or at least one B) = 1- P(no A and no B)
    \end{itemize}
    \section[Independence]{Independence}
    \paragraph[Definition]{Definition: } Events A and B are independent if
    $P(A\bigcapB) = P(A)P(B)$
    \paragraph[Example]{Example: }Flip a fair coin twice. Then
    $S = \{HH, HT, TH, TT\}$
    \begin{itemize}
        \item A = $1^{st}$ toss is heads $\to A= \{HH, HT\}$ P(A) = 1/2
        \item B = $2^{nd}$ toss is heads $\to B= \{HH, TH\}$ P(B) = 1/2
        \item C = both tosses are heads $\to C= \{HH\}$ P(C) = 1/4
    \end{itemize}
    Then:
    \begin{itemize}
        \item A and B are independent:
        $P(A\bigcap B) = \frac{1}{4} = P(A)P(B) = \frac{1}{2}* \frac{1}{2} = \frac{1}{4}$
        \item A and C are not indepndent:
        $P(A\bigcap C) = \frac{1}{4} \neq P(A)P(C) = \frac{1}{2} * \frac{1}{4}$
    \end{itemize}
    \part[Random Variables and Mass Density Functions]{Random Variables and Mass Density Functions}
    \paragraph[Definition]{Definition: } A random variable, X, is a function
    that assigns a real number to each element of the
    sample space. The \underline{mass density function} of X
    is the function defined by
    $p(k) = P(x=k)$

    \paragraph[Example]{Example: } Flip a fair coin twice. Let X be
    the \# of heads.

    \begin{align*}
        HH \to 2 && HT \to 1 && TH \to 1 && TT \to 0
    \end{align*}
    The range of X is \{0, 1, 2\}. The mass density function of
    X is:
    \begin{itemize}
        \item $p(0) = \frac{1}{4} = P(x=0)$
        \item $p(1) = \frac{2}{4} = P(x=1)$
        \item $p(2) = \frac{1}{4} = P(x=2)$
    \end{itemize}
    \paragraph[Example]{Example: } Roll two six-sided dice.
    Let $Y$ equal the sum of all the rolls.
    Find the mass density function of Y.
    \begin{itemize}
        \item The range of Y is \{2, 3, 4, ..., 12\}
        \subitem $p(2) = \frac{1}{36} = p(12)$
        \subitem $p(3) = \frac{2}{36} =p(11)$
        \subitem $p(4) = \frac{3}{36} = p(10)$
        \subitem $p(5) = \frac{4}{36}= p(9)$
        \subitem $p(6) = \frac{5}{36} = p(8)$
        \subitem $p(7) = \frac{6}{36}$

        $ \sum_{k=2}^{12} p(k) = 1$
    \end{itemize}
    \part[Expected Value]{Expected Value}
    Supposed a group of students has the following 7 quiz scores:

    3, 5, 8, 8, 8, 10, 10

    The average quiz score is:

    $\frac{3+5+8+8+8+10+10}{7} = \frac{52}{7}$

    Alternatively, we could find the average score by using
    a weighted average:

    $3(\frac{1}{7}) + 5(\frac{1}{7}) + 8(\frac{3}{7} + 10(\frac{2}{7}) = \frac{52}{7})$

    \paragraph[Definition]{Definition: } Suppose random variable X takes on values
    $x_1, x_2, ..., x_n$ Then the expected value of X is
    \begin{align*}
        E[X] = \sum_{k=1}^{n} x_k P(X=x_k) = \sum_{k=1}^{n} x_{k}p(x_k)
    \end{align*}

    \paragraph[Example]{Example: } Flip a fair coin 3 times.
    be the \# of heads. Find $E[X]$
    \begin{align*}
        S = \{HHH, HHT, HTH, THH, TTH, THT, HTT, TTT \}
    \end{align*}
    \begin{itemize}
        \item range of X is $\{0,1,2,3\}$
        \item mass density function:
        \subitem $p(0) = \frac{1}{8}$
        \subitem $p(1) = \frac{3}{8}$
        \subitem $p(2) = \frac{3}{8}$
        \subitem $p(3) = \frac{1}{8}$
    \end{itemize}
    \begin{align*}
        E[X] &= (0)+p(0) + (1)+p(1) + (2)p(2) + (3)p(3) \\
        &= (1)\frac{3}{8} + (2)\frac{3}{8} + (3)\frac{1}{8} \\
        &= \frac{1}{8}(3+6+3) = \frac{12}{8} = \frac{3}{2}
    \end{align*}

    \part[Expected Value Examples]{Expected Value Examples}
    \paragraph[Example]{Example: } Flip a fair coin 3 times.
    Let$Y = #$ of heads multiplied by the # of tails, Find $E[Y]$
    \begin{itemize}
        \item range of $Y= \{0, 2\}$
        \item mass density function
        \subitem $p(0) = 2/8 = 1/4$
        \subitem $p(2) = 6/8 = 3/4$
    \end{itemize}
    $E[Y] = 0p(0) + 2p(2) = 2(\frac{3}{4} = \frac{3}{2})$

    \paragraph[Example]{Example: } Supposed you play a game where you roll two six-sided dice.
    If the sum of the rolls is $> 9$, you win two dollars.
    Otherwise you lose $c$.
    Find the $c$ so that this is a fair game.
    Let $X$ be your winnings.
    \begin{itemize}
        \item The range of $X = \{2, -c\}$ Either win two dollars are lose c dollars
        \subitem $p(2) = \frac{3}{36} + \frac{2}{36} + \frac{1}{36} = \frac{1}{6}$
        \subitem $p(-c) = \frac{5}{6}$ Complement rule
    \end{itemize}
    $E[X] = 2p(2) + (-c)p(c) = 2(\frac{1}{6}) - c(\frac{5}{6})$
    $c = \$\frac{2}{5}$
    \paragraph[Example]{Example: } Supposed you keep flipping a pair of coints until you see HH.
    Let $X$ be the \# of times you need to toss the pair for this to occur
    \begin{itemize}
        \item What is the range of X?
        \subitem $\{1,2,3,4, ...\} \to$ infinitely many values of X
        \item Find the mass density function
        \subitem $p(1) = \frac{1}{4}$
        \subitem $p(2) = \frac{3}{4} \frac{1}{4}$
        \subitem $p(3) = \frac{3}{4} \frac{3}{4} \frac{1}{4} = \frac{3}{4}^2(\frac{1}{4})$
        \subitem $p(k) = (\frac{3}{4})^{k-1}(\frac{1}{4})$
    \end{itemize}
    We know
    $ \sum_{k=1}^{\infty} p(k) =1$ (From probability principles)
    This means
    \begin{align*}
        \sum_{k=1}^{\infty} (\frac{3}{4})^{k-1}(\frac{1}{4}) = 1.
    \end{align*}
    How would we know this without the context of probability?
\end{document}
